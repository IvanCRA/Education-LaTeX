\documentclass[a4paper, 11pt]{article}
\usepackage{graphicx} % Required for inserting images
\usepackage[utf8x]{inputenc}
\usepackage[english,russian]{babel}
\usepackage{cmap}
\usepackage{amsmath}
\usepackage{adjustbox}  % Для обрезки и трансформаций изображений
\usepackage{multirow}   % Для объединения строк в таблице
\usepackage{booktabs}
\usepackage[left = 3cm, right = 3cm, top = 3cm, bottom = 3cm]{geometry}

\title{Выполнение задания №2}
\author{Ivan Pertrischev}
\date{March 2025}

\begin{document}

\maketitle

\section{Практика, Вариант 9}
\section{Математический анализ}

\section*{Задание 1: Вставка диаграммы}
\begin{center}
    \includegraphics[width=10cm,height=6cm,trim=0cm 2cm 0cm 0cm,clip,angle=12]{chart.jpg}
\end{center}

\section*{Задание 2: Таблица с объединением строк}
\begin{table}[h]
    \centering
    \begin{tabular}{ccc}
        \hline
        \multirow{2}{*}{\textbf{Центр}} & \textbf{Колонка 2} & \textbf{Колонка 3} \\
        \cline{2-3}
        & Ещё & Данные \\
        \hline
    \end{tabular}
\end{table}

\section*{Задание 3: Плавающая таблица}
\begin{table}[b]
    \centering
    \resizebox{0.7\textwidth}{!}{ % Обрезка ширины таблицы
    \begin{tabular}{c c c}
        \toprule
        A & B & C \\
        \midrule
        1 & 2 & 3 \\
        4 & 5 & 6 \\
        \bottomrule
    \end{tabular}
    }
    \vspace{5pt} 
    \footnotesize Примечание: Здесь можно написать пояснение к таблице.
\end{table}

\end{document}