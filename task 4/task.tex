\documentclass{beamer}
\usetheme{Madrid} % можно поменять на Warsaw, CambridgeUS, и т.п.
\usecolortheme{beaver} % цветовая схема
\usefonttheme{default}

\usepackage[utf8]{inputenc}
\usepackage[russian]{babel}
\usepackage{graphicx}
\usepackage{listings}
\usepackage{booktabs} % для таблиц

\title{Пример презентации}
\author{Вано}
\institute{Университет Технологий}
\date{Июнь 2025}

\begin{document}

% Титульный слайд
\begin{frame}
  \titlepage
\end{frame}

% Списки
\begin{frame}{Списки}
\begin{itemize}
  \item Первый пункт
  \item Второй пункт
  \begin{itemize}
    \item Подпункт
  \end{itemize}
\end{itemize}
\end{frame}

% Таблица
\begin{frame}{Таблица}
\begin{tabular}{@{}ll@{}}
\toprule
Название & Значение \\
\midrule
A        & 100      \\
B        & 200      \\
\bottomrule
\end{tabular}
\end{frame}

% Картинка
\begin{frame}{Картинка}
\begin{center}
  \includegraphics[width=0.6\textwidth]{image.png}
\end{center}
\end{frame}

% Исходный код
\begin{frame}[fragile]{Исходный код на C++}
\begin{lstlisting}[language=C++]
#include <iostream>
int main() {
    std::cout << "Hello world!";
    return 0;
}
\end{lstlisting}
\end{frame}

% Блок с рамкой
\begin{frame}{Важный блок}
\begin{block}{Важно!}
Это важная информация, выделенная рамкой.
\end{block}
\end{frame}

% Заметка (цитата)
\begin{frame}{Заметка}
\begin{quote}
Это блок заметки. Здесь можно писать пояснения или интересные факты.
\end{quote}
\end{frame}

% Финал
\begin{frame}{Спасибо!}
Спасибо за внимание!
\end{frame}

\end{document}