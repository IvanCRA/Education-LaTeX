\documentclass[a4paper,12pt]{article}
\usepackage[T2A]{fontenc}
\usepackage[utf8]{inputenc}
\usepackage[russian]{babel}
\usepackage{amsmath}
\usepackage{amssymb}
\usepackage{indentfirst}

\title{Конспект: Микроконтроллеры AVR}
\author{}
\date{}

\begin{document}

\maketitle

\section{Структура AVR}
Микроконтроллеры AVR имеют гарвардскую архитектуру с раздельными шинами для команд и данных. Основные компоненты:
\begin{itemize}
    \item 8-битное RISC-ядро
    \item Флэш-память программ
    \item ОЗУ (SRAM) для данных
    \item Энергонезависимая EEPROM
    \item Периферийные устройства (таймеры, АЦП, USART и др.)
\end{itemize}

\section{Память}
\subsection{Организация памяти}
\begin{itemize}
    \item \textbf{Память программ (Flash)} - до 256 КБ, 16-битные ячейки
    \item \textbf{ОЗУ (SRAM)} - до 16 КБ, 8-битные ячейки
    \item \textbf{EEPROM} - до 4 КБ, 8-битные ячейки, 100 000 циклов записи
\end{itemize}

\subsection{Регистры}
\begin{itemize}
    \item 32 \textbf{регистра общего назначения} (R0-R31)
    \item \textbf{Регистр статуса} (SREG) - флаги процессора
    \item \textbf{Указатель стека} (SP) - 16-битный
    \item \textbf{Регистры ввода/вывода} (I/O Registers)
\end{itemize}

\section{Система команд}
Типы инструкций:
\begin{itemize}
    \item Арифметические и логические (ADD, SUB, AND, OR, INC, DEC)
    \item Операции с битами (SET, CLR, SBI, CBI)
    \item Переходы и вызовы (RJMP, IJMP, CALL, RET)
    \item Перемещение данных (MOV, LDS, STS, LDI)
    \item Управление (NOP, SLEEP, WDR)
\end{itemize}

Большинство инструкций выполняется за 1 такт.

\section{Способы адресации}
\begin{enumerate}
    \item \textbf{Прямая} - операнд в регистре (ADD R1, R2)
    \item \textbf{Непосредственная} - операнд в инструкции (LDI R16, 0xFF)
    \item \textbf{Косвенная} - через указатель (LD R1, X+)
    \item \textbf{Относительная} - для переходов (RJMP label)
    \item \textbf{Прямая адресация данных} - (LDS R16, 0x100)
    \item \textbf{Адресация ввода/вывода} - (IN R16, PORTB)
\end{enumerate}

\section{Прерывания}
\subsection{Типы прерываний}
\begin{itemize}
    \item Внешние (INT0, INT1)
    \item От таймеров
    \item От USART
    \\item От АЦП
    \item Аппаратный сброс
\end{itemize}

\subsection{Обработка прерываний}
\begin{enumerate}
    \item Завершение текущей инструкции
    \item Сохранение PC в стеке
    \item Сброс флага прерывания
    \item Переход по вектору прерывания
    \item Выполнение обработчика
    \item Восстановление PC (RETI)
\end{enumerate}

Вектора прерываний расположены в начале Flash-памяти.

\end{document}