\documentclass[a4paper,12pt]{article}
\usepackage[T2A]{fontenc}
\usepackage[utf8]{inputenc}
\usepackage[russian]{babel}
\usepackage{amsmath}
\usepackage{amssymb}
\usepackage{indentfirst}

\title{Конспект: Микроконтроллеры AVR}
\author{}
\date{}

\begin{document}

\maketitle

\section{Структура AVR}
Микроконтроллеры AVR имеют гарвардскую архитектуру с раздельными шинами для команд и данных. Это обеспечивает параллельную загрузку инструкций и данных, увеличивая производительность.

Основные компоненты:
\begin{itemize}
    \item 8-битное RISC-ядро с набором оптимизированных инструкций
    \item Flash-память программ (перепрограммируемая)
    \item Оперативная память (SRAM)
    \item Постоянная память (EEPROM)
    \item Таймеры/счётчики
    \item Аналого-цифровой преобразователь (АЦП)
    \item Последовательный интерфейс (USART, SPI, I2C)
    \item Прерывания и система управления питанием
\end{itemize}

\section{Память}
\subsection{Организация памяти}
Память микроконтроллера делится на три основные области:
\begin{itemize}
    \item \textbf{Flash} — энергонезависимая память программ (до 256 КБ), хранит исполняемый код. Доступ осуществляется 16-битными словами.
    \item \textbf{SRAM} — оперативная память для переменных и стека. Обычно до 16 КБ. Ячейки 8-битные.
    \item \textbf{EEPROM} — энергонезависимая память для хранения пользовательских данных. Поддерживает около $10^5$ циклов записи.
\end{itemize}

\subsection{Регистры}
\begin{itemize}
    \item 32 \textbf{регистра общего назначения} (R0–R31), используемые для большинства операций.
    \item \textbf{Регистр статуса} (SREG) содержит 8 флагов, отображающих состояние ALU (арифметико-логического устройства).
    \item \textbf{Указатель стека (SP)} — хранит адрес вершины стека, 16-битный.
    \item \textbf{I/O-регистры} — специальные регистры для управления периферией.
\end{itemize}

\subsection{Флаги регистра SREG}
\begin{itemize}
    \item \textbf{I} — глобальное разрешение прерываний
    \item \textbf{T} — временный бит (Transfer Bit)
    \item \textbf{H} — половинный перенос (Half Carry)
    \item \textbf{S} — знак (Sign = N $\oplus$ V)
    \item \textbf{V} — переполнение (Overflow)
    \item \textbf{N} — отрицательный результат
    \item \textbf{Z} — результат равен нулю
    \item \textbf{C} — перенос (Carry)
\end{itemize}

\section{Система команд}
AVR использует компактный и эффективный набор инструкций. Большинство инструкций исполняется за 1 такт.

Типы инструкций:
\begin{itemize}
    \item Арифметика и логика: \texttt{ADD, SUB, AND, OR, INC, DEC}
    \item Операции с битами: \texttt{SET, CLR, SBI, CBI}
    \item Переходы: \texttt{RJMP, IJMP, CALL, RET}
    \item Работа с памятью: \texttt{MOV, LDS, STS, LDI, PUSH, POP}
    \item Управление системой: \texttt{NOP, SLEEP, WDR, SEI, CLI}
\end{itemize}

\section{Способы адресации}
\begin{enumerate}
    \item \textbf{Прямая} — указание регистров: \texttt{ADD R1, R2}
    \item \textbf{Непосредственная} — значение закодировано в инструкции: \texttt{LDI R16, 0xFF}
    \item \textbf{Косвенная} — через регистры X, Y, Z: \texttt{LD R0, X+}
    \item \textbf{Относительная} — используется в переходах: \texttt{RJMP LABEL}
    \item \textbf{Прямая адресация SRAM} — через абсолютный адрес: \texttt{LDS R16, 0x100}
    \item \textbf{Адресация I/O} — доступ к портам и регистрам: \texttt{IN R16, PORTB}
\end{enumerate}

\section{Прерывания}
\subsection{Типы прерываний}
\begin{itemize}
    \item Внешние: \texttt{INT0}, \texttt{INT1}
    \item От таймеров/счётчиков (переполнение, сравнение)
    \item От USART (приём/передача данных)
    \item От АЦП (завершение преобразования)
    \item От SPI/I2C
    \item Аппаратный сброс и Watchdog
\end{itemize}

\subsection{Обработка прерываний}
\begin{enumerate}
    \item Завершение текущей инструкции
    \item Сохранение адреса возврата (PC) в стек
    \item Переход к обработчику прерывания (ISR)
    \item Выполнение обработчика
    \item Команда \texttt{RETI} восстанавливает PC
\end{enumerate}

Вектора прерываний расположены по фиксированным адресам в начале Flash-памяти.

\section{Таймеры/Счётчики}
AVR содержит до трёх таймеров:
\begin{itemize}
    \item 8-битные и 16-битные таймеры (Timer0, Timer1, Timer2)
    \item Поддержка режима счёта, сравнения, PWM
    \item Возможность генерации прерываний по совпадению, переполнению
    \item Используются для временных задержек, ШИМ, и периодических событий
\end{itemize}

\section{Аналого-цифровой преобразователь (АЦП)}
\begin{itemize}
    \item 10-битный АЦП (в большинстве AVR)
    \item Поддержка до 8 аналоговых входов (в зависимости от модели)
    \item Настраиваемое опорное напряжение (AVCC, AREF, внутреннее)
    \item Возможность автоматического запуска по триггеру
    \item Генерация прерывания по завершении преобразования
\end{itemize}

\section{Архитектура шин}
AVR использует модифицированную гарвардскую архитектуру, где:
\begin{itemize}
    \item Команды и данные передаются по разным шинам — это позволяет загружать новую инструкцию, пока текущая выполняется.
    \item Доступ к Flash-памяти осуществляется отдельной шиной команд.
    \item SRAM и I/O устройства доступны через шину данных.
\end{itemize}
Это повышает производительность и позволяет большинству команд выполняться за 1 такт.

\section{Стек}
\begin{itemize}
    \item Стек в AVR размещён в SRAM и управляется 16-битным указателем стека (SP).
    \item Используется для сохранения адреса возврата при вызовах подпрограмм и обработке прерываний.
    \item Команды \texttt{PUSH} и \texttt{POP} — явное управление стеком.
\end{itemize}

\section{Ввод/вывод (I/O)}
Работа с портами осуществляется через три регистра:
\begin{itemize}
    \item \texttt{DDRx} — направление (1 — выход, 0 — вход)
    \item \texttt{PORTx} — установка значения (для выхода) или подтягивание (для входа)
    \item \texttt{PINx} — чтение значения на входе
\end{itemize}

\textbf{Пример:} установка PORTB0 как выхода и подача лог. 1
\begin{verbatim}
    sbi DDRB, 0   ; PORTB0 как выход
    sbi PORTB, 0  ; установить лог. 1 на PORTB0
\end{verbatim}

\section{Тактирование}
\begin{itemize}
    \item Варианты тактовых источников: внутренний RC-генератор, внешний кварц, внешний источник.
    \item Частота может достигать до 20 МГц (в зависимости от модели).
    \item Делители частоты (prescalers) используются в таймерах и ADC.
\end{itemize}

\section{Управление питанием}
AVR поддерживает несколько режимов энергосбережения:
\begin{itemize}
    \item \texttt{Idle} — остановка ЦП, но работают прерывания и периферия
    \item \texttt{Power-down} — почти всё отключено, минимальное потребление
    \item \texttt{ADC Noise Reduction} — минимизация помех при работе АЦП
\end{itemize}
Переход осуществляется через команду \texttt{SLEEP}, а выход — по прерыванию.

\section{Арифметико-логическое устройство (ALU)}
ALU выполняет основные арифметические и логические операции:
\begin{itemize}
    \item Сложение, вычитание (с флагами переноса и заёма)
    \item Побитовые операции (AND, OR, XOR, NOT)
    \item Сдвиги (LSL, LSR, ASR, ROR)
    \item Управление флагами регистра SREG
\end{itemize}
ALU работает с регистрами общего назначения (R0–R31), в большинстве операций участвуют только они.

\section{Пример простой программы на ассемблере}
\textbf{Программа мигания светодиодом на порту B0:}
\begin{verbatim}
    ldi r16, 0x01        ; установить бит 0
    out DDRB, r16        ; порт B0 как выход
loop:
    out PORTB, r16       ; включить светодиод
    rcall delay
    out PORTB, r1        ; выключить
    rcall delay
    rjmp loop

delay:                  ; простая задержка
    ldi r18, 100
wait1:
    ldi r19, 255
wait2:
    dec r19
    brne wait2
    dec r18
    brne wait1
    ret
\end{verbatim}

\section{Типы микроконтроллеров AVR}
Линейка AVR включает несколько семейств:
\begin{itemize}
    \item \textbf{tinyAVR} — простые, маломощные (например, ATtiny85)
    \item \textbf{megaAVR} — более функциональные, с большим числом портов (например, ATmega328P)
    \item \textbf{XMEGA} — более мощные, с улучшенным управлением питанием, DMA, повышенной частотой
\end{itemize}

\section{Системы сброса (reset)}
AVR поддерживает несколько источников сброса:
\begin{itemize}
    \item \textbf{Аппаратный (external)} — по входу \texttt{RESET}
    \item \textbf{Watchdog reset} — при переполнении таймера сторожевого таймера
    \item \textbf{Power-on reset} — при включении питания
    \item \textbf{Brown-out reset} — при просадке питания ниже допустимого уровня
\end{itemize}
Состояние сброса можно отследить через регистр \texttt{MCUSR}.

\section{Программаторы и прошивка}
AVR поддерживает несколько способов прошивки:
\begin{itemize}
    \item \textbf{ISP (In-System Programming)} — наиболее распространённый способ (через SPI)
    \item \textbf{PDI, TPI} — у новейших моделей
    \item \textbf{JTAG} — поддерживается для отладки и прошивки старших моделей
\end{itemize}
Типовые программаторы:
\begin{itemize}
    \item USBasp (самодельный/китайский)
    \item AVRISP mkII
    \item Arduino как программатор (через ArduinoISP)
\end{itemize}

\section{Сравнение с другими архитектурами}
AVR по сравнению с другими МК:
\begin{itemize}
    \item Простая архитектура, легко изучать
    \item Интуитивно понятные регистры и I/O
    \item По сравнению с ARM — медленнее, меньше памяти, но проще
    \item По сравнению с PIC — чище архитектура и более дружелюбная к C/ASM разработке
\end{itemize}
AVR идеально подходит для учебных целей, прототипирования и простых проектов.


\end{document}
